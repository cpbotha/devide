\documentstyle[a4,makeidx,verbatim,texhelp,fancyhea,mysober,mytitle]{report}%
%\input{psbox.tex}
\newcommand{\commandref}[2]{\helpref{{\tt $\backslash$#1}}{#2}}%
\newcommand{\commandrefn}[2]{\helprefn{{\tt $\backslash$#1}}{#2}\index{#1}}%
\newcommand{\commandpageref}[2]{\latexignore{\helprefn{{\tt $\backslash$#1}}{#2}}\latexonly{{\tt $\backslash$#1} {\it page \pageref{#2}}}\index{#1}}%
\newcommand{\indexit}[1]{#1\index{#1}}%
\newcommand{\inioption}[1]{{\bf {\tt #1}}\index{#1}}%
\parskip=10pt%
\parindent=0pt%
%\backgroundcolour{255;255;255}\textcolour{0;0;0}% Has an effect in HTML only
\winhelpignore{\title{DeVIDE Help}

\author{Charl P. Botha}%
}%
\winhelponly{\title{DeVIDE Help}

\author{Charl P. Botha}%
}%
\makeindex%
\begin{document}%
\maketitle%
\pagestyle{fancyplain}%
\bibliographystyle{plain}%
\pagenumbering{roman}%
\setheader{{\it CONTENTS}}{}{}{}{}{{\it CONTENTS}}%
\setfooter{\thepage}{}{}{}{}{\thepage}%
\tableofcontents%

%%%%%%%%%%%%%%%%%%%%%%%%%%%%%%%%%%%%%%%%%%%%%%%%%%%%%%%%%%%%%%%%%%%%%%%%%%%
\chapter*{Copyright notice}%
\setheader{{\it COPYRIGHT}}{}{}{}{}{{\it COPYRIGHT}}%
\setfooter{\thepage}{}{}{}{}{\thepage}%
DeVIDE is copyright (c) 2004 by Charl P. Botha <cpbotha@ieee.org>

This software is licensed exclusively for research use by authorised
parties.  All unauthorised use or distribution is strictly prohibited.

Any modifications made to this software, with exception of work in the
userModules/ tree, shall be sent to the author for possible inclusion
in future versions.  Ownership and copyright of said modifications
shall be unconditionally ceded to the author.

THIS SOFTWARE IS PROVIDED BY THE COPYRIGHT HOLDERS AND CONTRIBUTORS
``AS IS'' AND ANY EXPRESS OR IMPLIED WARRANTIES, INCLUDING, BUT NOT
LIMITED TO, THE IMPLIED WARRANTIES OF MERCHANTABILITY AND FITNESS FOR
A PARTICULAR PURPOSE ARE DISCLAIMED.  IN NO EVENT SHALL THE AUTHORS OR
CONTRIBUTORS BE LIABLE FOR ANY DIRECT, INDIRECT, INCIDENTAL, SPECIAL,
EXEMPLARY, OR CONSEQUENTIAL DAMAGES (INCLUDING, BUT NOT LIMITED TO,
PROCUREMENT OF SUBSTITUTE GOODS OR SERVICES; LOSS OF USE, DATA, OR
PROFITS; OR BUSINESS INTERRUPTION) HOWEVER CAUSED AND ON ANY THEORY OF
LIABILITY, WHETHER IN CONTRACT, STRICT LIABILITY, OR TORT (INCLUDING
NEGLIGENCE OR OTHERWISE) ARISING IN ANY WAY OUT OF THE USE OF THIS
SOFTWARE, EVEN IF ADVISED OF THE POSSIBILITY OF SUCH DAMAGE.

%%%%%%%%%%%%%%%%%%%%%%%%%%%%%%%%%%%%%%%%%%%%%%%%%%%%%%%%%%%%%%%%%%%%%%%%%%%
\chapter{Introduction}%
\pagenumbering{arabic}%
\setheader{{\it CHAPTER \thechapter}}{}{}{}{}{{\it CHAPTER \thechapter}}%
\setfooter{\thepage}{}{}{}{}{\thepage}%
This is the user manual and help file for the DeVIDE software package.
DeVIDE is the Delft Visualisation and Image processing Development
Environment.

%%%%%%%%%%%%%%%%%%%%%%%%%%%%%%%%%%%%%%%%%%%%%%%%%%%%%%%%%%%%%%%%%%%%%%%%%%%
\chapter{Graph Editor}
\pagenumbering{arabic}%
\setheader{{\it CHAPTER \thechapter}}{}{}{}{}{{\it CHAPTER \thechapter}}%
\setfooter{\thepage}{}{}{}{}{\thepage}%

The DeVIDE Graph Editor is a visual programming interface where glyphs
representing the underlying DeVIDE modules can be connected together
to form new programs.  It's the most flexible way of working with
DeVIDE, short of directly interfacing with the underlying code.

This chapter will give a brief overview of graph editor usage.

\section{A few important tips}
Before we start, a few productivity-enhancing secrets of the Graph
Editor are revealed here.

\subsection{Getting module-specific help}
Many modules have module-specific help-text built in.  To see this
help-text, create the relevant module-glyph on the canvas, right click
on the module, and select ``Help on Module''.

Also remember that selected (special) modules are documented in the
chapter \helpref{Special Modules}{secSpecialModules}.

\subsection{Reading data, quickly}
Dragging and dropping certain data-files on the Graph Editor canvas
will cause the system to automatically create and configure the
applicable module.  For example: selecting, dragging and dropping a
collection of DICOM .dcm files on the canvas will result in a dicomRDR
glyph to be created and pre-configured with the list of files that has
been dropped.

\subsection{Re-using networks, quickly}
Dragging and dropping a .dvn (DeVIDE Network) file on the canvas will
instantly load the network and build it at the mouse position.  This
will not destroy any of your current networks on the canvas.

\subsection{Quick-type module find}
Select the module categories that you want to work with.  Now focus
the canvas by clicking on it with your mouse.  Modules can now be
quickly selected by typing the first few letters in their names.  Once
the module you want is highlighted, press the ``Enter'' key to place
it.

\section{A small sample network}
The Graph Editor is explained by putting together a simple network for
the creating of a volume and the subsequent extraction of an
iso-surface from the synthesised volume.

%%%%%%%%%%%%%%%%%%%%%%%%%%%%%%%%%%%%%%%%%%%%%%%%%%%%%%%%%%%%%%%%%%%%%%%%%%%
\chapter{Special modules}\label{secSpecialModules}
\pagenumbering{arabic}%
\setheader{{\it CHAPTER \thechapter}}{}{}{}{}{{\it CHAPTER \thechapter}}%
\setfooter{\thepage}{}{}{}{}{\thepage}%



\section{The slice3dVWR}

%\bibliography{refs}
%\addcontentsline{toc}{chapter}{Bibliography}
%\setheader{{\it REFERENCES}}{}{}{}{}{{\it REFERENCES}}%
%\setfooter{\thepage}{}{}{}{}{\thepage}%

\addcontentsline{toc}{chapter}{Index}
\setheader{{\it INDEX}}{}{}{}{}{{\it INDEX}}%
\setfooter{\thepage}{}{}{}{}{\thepage}%
\printindex%

\end{document}
